\documentclass[12pt]{article}
\usepackage[T1]{fontenc}
\usepackage[utf8]{inputenc}
\usepackage[estonian]{babel}
\usepackage{longtable}
\usepackage{adjustbox}
\usepackage{enumitem}
\usepackage{subcaption}
\usepackage{amsmath}
\usepackage{amssymb}
\usepackage{amsthm}
\usepackage{numprint}
\usepackage{dirtytalk}
\usepackage{float}
\usepackage{soul}
\usepackage{enumitem}
\usepackage[pdftex]{graphicx}
\usepackage{titling}
\usepackage{eurosym}
\usepackage{marvosym}
\usepackage{url}
\usepackage[bookmarksopen, bookmarksdepth=2, breaklinks=true]{hyperref}
\def\UrlBreaks{\do\/\do-}
\usepackage{hyperref}
\hypersetup{
    colorlinks=true,
    linkcolor=blue,
    filecolor=magenta,
    urlcolor=cyan,
}
\setlength{\voffset}{-0.75in}
\setlength{\textheight}{650pt}
\setlength{\textwidth}{476pt}
\setlength{\hoffset}{-0.75in}

\title{Growth of a population}
\author{Meelis Utt}
\date{September 2021}

\begin{document}

\maketitle

\section*{Description}
In a small town the population is $P_0$ at the beginning of a year.
The population regularly increases by $percent\%$ per year and moreover $aug$ new inhabitants per year come to live in the town.
How many years $N$ does the town need to see its population greater or equal to $P_N$ inhabitants?
Arguments $P_0$, $P_N$, $percent$ and $aug$ are known.

\subsection*{Derivation}
First, lets denote
\begin{equation*}
    r = \bigg(1+\frac{percent}{100}\bigg)
\end{equation*}
To can use the following formulas to find the population at $N$th year:
\begin{align}
    \nonumber P_1 &= rP_0 + aug\\
    \nonumber P_2 &= rP_1 + aug \Rightarrow P_2 = r^{2}P_0 + (r + 1)aug\\
    \nonumber P_3 &= rP_2 + aug \Rightarrow P_3 = r^{3}P_0 + (r^{2} + r + 1)aug\\
    \nonumber P_4 &= rP_3 + aug \Rightarrow P_4 = r^{4}P_0 + (r^{3} + r^{2} + r + 1)aug\\
    \nonumber &...\\
    P_N &= r^{N}P_0 + \sum_{i=0}^{N-1}r^{i}aug \label{growth-fun}
\end{align}

We have two cases we need to handle:
\begin{enumerate}
    \item when growth percent is equal to $0$, i.e. $percent = 0$;
    \item when growth percent is not equal to $0$, i.e. $percent \ne 0$.
\end{enumerate}

\hspace{-0.5}When $percent = 0$, then
\begin{equation*}
    P_N = P_0+N\cdot aug.
\end{equation*}
When $percent \ne 0$, then we notice, that for variable $aug$ a geometric series $[\ref{geom}]$ forms.
We know, that when $r\ne 1$, then
\begin{equation}
    \sum_{i=0}^{N-1}r^{i}aug = aug\frac{1-r^{N}}{1-r}. \label{geom-series}
\end{equation}
Since $r$ is always $r>1$ in our case, then we can write $(\ref{geom-series})$ as
\begin{equation}
    aug\frac{r^{N}-1}{r-1}. \label{geom-series-inv}
\end{equation}
Substituting $(\ref{geom-series})$ and $(\ref{geom-series-inv})$ into $(\ref{growth-fun})$ we get
\begin{align*}
    P_{N} &= r^{N}P_{0} + aug\frac{r^{N}-1}{r-1}\\
    P_{N} &= \frac{r^{N}(r-1)P_{0} + aug\cdot r^{N}-aug}{r-1}\\
    P_{N} &= \frac{r^{N}\big[(r-1)P_{0} + aug\big]-aug}{r-1}\\
    &\Rightarrow r^{N} = \frac{(r-1)P_{N}+aug}{(r-1)P_{0} + aug} \bigg | \log\\
    &\Rightarrow N\log(r) = \log\bigg(\frac{(r-1)P_{N}+aug}{(r-1)P_{0} + aug}\bigg)\\
    &\Rightarrow N = \frac{\log\bigg(\frac{(r-1)P_{N}+aug}{(r-1)P_{0} + aug}\bigg)}{\log(r)}.
    \qed
\end{align*}

\section*{Cites}
\begin{enumerate}
    \item \url{https://en.wikipedia.org/wiki/Geometric_series} \label{geom}
\end{enumerate}

\end{document}
